\section{Conclusion}
\label{sec:conclusion}
In recent years, the growth of artificial intelligence and deep learning has necessitated improvements to data engineering systems to handle the increasing size and complexity of data. We have introduced Cylon, a high-performance C++ library that supports ETL pipelines in Python. Previous papers have highlighted Cylon's superior performance compared to other similar Python data engineering frameworks, such as Abeykook et al.'s paper, "Data Engineering for HPC with Python."  

In this work, we present our integration of UCX and UCC as a replacement for MPI in BSP communication between processes. The novelty of this work is apparent based on a survey of the literature on similar work involving distributed data frames. We applied this work directly and refined the initial implementation to support both serverful and serverless architectures of cloud providers like Amazon Web Services. Additionally, we implemented direct support for Python via Cython, enabling the execution of data engineering tasks in Python without the need to write native code. We also completed the integration of UCC into Cylon by removing our implementation of \textit{AllGatherV} in favor of UCC's support for variable-length data. Furthermore, we implemented container support across infrastructures to facilitate the execution of experiments and library usage.

During the course of this research we encountered a number of challenges that resulted in delays.  We first validated running Cylon on AWS by building the library using AWS Workspaces which was generally successful.   Using AWS facilitated building and running on bare metal because of full control of the underlying ECS host.  Comparatively, building on bare metal on Rivanna was plagued with difficulties based on library changes and security imposed by system administrators for security purposes.  In a related work, RP-Cylon encountered similar delays in months based on challenges building Cylon on Rivanna.  That problem was ultimately solved by addressing fundamental problems with the build to install process.  Later we experienced a similar issue trying to build UCX Cylon after a period of year when updates were applied that broke the previously successful build process.  We were able to solve this problem by using Apptainer or Singularity and running Cylon in a container.  It is our view that use of Apptainer or Docker containers is the future for Cylon based on the inherent challenges associated with building native code on constrained hardware clusters such as Rivanna.

A second problem detailed elsewhere in this paper was related to implementing NAT hole punching within the UCX library.  It was our intention to run Cylon in serverless and serverful environments using UCC/UCX.  The idea was to modify the TCP capability in this library and implement a similar mechanism as what is implemented in the FMI library.  The reasoning behind this choice was FMI shared a similar architecture being developed in C++.  After many months of effort, we were not able to achieve this. 
 We learned and observed the semantics associated with endpoint communication establishment within UCX does not necessary follow the establishment specifics that is required by NAT Traversal techniques.  In retrospect, we believe a fully custom UCX transport would have been an achievable solution.  A key related challenge was the undocumented nature of the UCX library and the complexity of the C code within this library.  Lastly, the device discovery mechanism implemented by UCX uses system level calls to determine the most ideal and optimal communication hardware.  For serverless environments, access to these system calls is prohibited.  This led to the failures executing AWS Fargate experiments based on a change or changes implemented by AWS.

A third problem was related to AWS funding.  We were initially successful receiving funding, but unfortunately research funding expired before we were able to complete our work.  It is an ongoing challenge to perform research on AWS based on the unavailability of AWS as a service by the University and the implicit requirement to apply for funding in order to use cloud providers for research such as what is detailed in this paper.

In this paper, We demonstrated experiments using Cylon and UCC/UCX, comparing them to similar experiments conducted on the Rivanna HPC cluster, to validate the effectiveness and applicability of running data engineering tasks at scale. An emerging paradigm in cloud computing is the use of serverless infrastructure, allowing developers to concentrate on software or applications without the burden of managing infrastructure or other undifferentiated workloads that can be handled by SaaS. We also discuss the challenges of using serverless specifically in communication and present an approach that utilizes NAT traversal through NAT hole punching to enable direct communication between AWS Lambda functions. Additionally, we highlight the relevance of this work to serverless containers such as AWS Fargate and emphasize the significance of architecting communication layers at high levels of abstraction when integrating with serverless compute. We plan to finalize our integration of FMI in the near future and add support for libfabric and checkpoints to facilitate data-parallel execution that extends beyond the fifteen-minute boundary condition imposed by AWS for lambda functions.


