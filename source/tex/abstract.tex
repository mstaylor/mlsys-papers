\begin{abstract}
    \textbf{Data can be found everywhere, from health to human infrastructure to the surge of sensors to the proliferation of internet-linked devices. To meet this challenge, the data engineering domain has expanded monumentally in recent years in both research and industry.  Additionally, In recent years, the data engineering discipline has been dramatically impacted by Artificial Intelligence (AI) and Machine Learning (ML), which has resulted in research on the speed, performance, and optimization of such processes. Traditionally, data engineering, Machine Learning, and Artificial Intelligence workloads have been executed on large clusters in a data center environment.  This requires considerable investment in terms of both hardware and maintenance.  With the advent of the public cloud, it is now possible to run large applications across nodes without maintaining or owning hardware.  Serverless computing has emerged in cloud and open-source varieties to meet such needs.  This allows users of such systems to focus on the application code and take advantage of bundled CPUs and memory configuration without focusing primarily on the semantics of horizontal scalability and resource allocation. }

    \textbf{Serverless functions like AWS Lambda offer horizontal scaling and fine-grained billing.  However, when executing jobs or tasks on large datasets, users resort to low-cost or ephemeral storage options much slower than traditional HPC clusters.  To address this limitation, the FMI library has been developed and includes support for distributed collective operations such as direct, reduce, allreduce and scan using point-to-point communication. We detail specific results comparing Cylon and FMI on the AWS Cloud and UVA’s Rivanna supercomputer with the ultimate goal of developing a novel approach to deep learning training and inference on both cloud serverful and serverless environments and HPCs.   We will thoroughly explain FMI and Cylon's architecture and the execution process of Cylon tasks using Slurm, AWS ECS and AWS Step Functions. We show AWS Cylon and FMI achieve reasonable performance with minimal and constant overhead and compare these results to HPC clusters such as Rivanna Cylon.}

%The approach aims to excel in both scientific and engineering research HPC systems while demonstrating robust performance on cloud infrastructures. This dual capability fosters collaboration and innovation within the open-source scientific research community.

\end{abstract}
